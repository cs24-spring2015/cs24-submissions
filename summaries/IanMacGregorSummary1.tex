% This file was converted to LaTeX by Writer2LaTeX ver. 1.4
% see http://writer2latex.sourceforge.net for more info
\documentclass{article}
\usepackage[ascii]{inputenc}
\usepackage[T1]{fontenc}
\usepackage[english]{babel}
\usepackage{amsmath}
\usepackage{amssymb,amsfonts,textcomp}
\usepackage{array}
\usepackage{hhline}
\usepackage[top=1.25in, bottom=1.25in, left=1.25in, right=1.25in]{geometry}
\title{}
\linespread{2}
\begin{document}
Ian MacGregor

Instructor Cesar Torres

Computer Science 24

23 April 2015

\begin{center}
Research Seminar Summary 1
\end{center}

\ \ I attended two seminars: Human-Centered Computer Security (one hour) and Policing the Police: Researching Civilian Oversight Boards (two hours). 

\ \ The first, Human-Centered Computer Security, wanted to tackle the goal of both maintaining the highest levels of security as well as user friendliness. Tamara Denning of the University of Utah had done research in the security of implantable cardiac devices. She identified the oversight by manufacturers of implantable cardiac devices in maintaining high levels of security in the very impactful and deadly potentiality of some person hacking a wireless implantable cardiac device and inducing a cardiac fibrillation. 

\ \ She made the insight that all of these major issues could be fixed with small security clarifications, which include merely maintaining best practices measures. The changes would not require any major changes in the medical field; in fact, they'd make it easier for medical officials to do their job by streamlining the charging and programming process. 

\ \ She also made the insight that there are a lot of different ways to maintain an accessible password so that the implantable cardiac device would only be hackable by medical officials in emergency circumstances and otherwise. 

\ \ Most of the presentation was accessible and understandable, but there were a few terms which were a little verbose, but I was able to understand the concepts via context clues. 
\end{document}
