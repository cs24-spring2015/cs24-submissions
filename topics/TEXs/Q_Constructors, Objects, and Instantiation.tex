\documentclass[11pt, oneside]{article}   	% use "amsart" instead of "article" for AMSLaTeX format
\usepackage{geometry}                		% See geometry.pdf to learn the layout options. There are lots.
\geometry{letterpaper}                   		% ... or a4paper or a5paper or ... 
%\geometry{landscape}                		% Activate for for rotated page geometry
%\usepackage[parfill]{parskip}    		% Activate to begin paragraphs with an empty line rather than an indent
\usepackage{graphicx}				% Use pdf, png, jpg, or eps§ with pdflatex; use eps in DVI mode
								% TeX will automatically convert eps --> pdf in pdflatex		
\usepackage{amssymb}

\title{Questions on Constructors, Objects, and Instantiation.}
\author{Elmer, Ethan, Lakshay and Tommy}
\date{}							% Activate to display a given date or no date

\begin{document}
\maketitle
\section*{Question 1}
What is the relationship between classes and objects?

\section*{Question 2}
What is a constructor? What is the purpose of a constructor in a class?

\section*{Question 3}
For this problem, you should write a very simple but complete class. The class represents a counter that counts 0, 1, 2, 3, 4,.... The name of the class should be Counter. It has one private instance variable representing the value of the counter. It has two instance methods: increment() adds one to the counter value, and getValue() returns the current counter value. Write a complete definition for the class, Counter.

\section*{Question 4}
\par
This problem uses the Counter class from Question 3. The following program segment is meant to simulate tossing a coin 100 times. It should use two Counter objects, headCount and tailCount, to count the number of heads and the number of tails. Fill in the blanks so that it will do so.\\ 

Counter headCount, tailCount;

tailCount = new Counter();

headCount = new Counter();

for ( int flip = 0;  flip \textless 100;  flip++ ) \{

\hspace{4ex} if (Math.random() \textless 0.5) \{   // There's a 50/50 chance that this is true.
             
\hspace{8ex} \textit{Your Code Here} ;   // Count a "head".

\hspace{4ex} \}
                 
\hspace{4ex} else \{
             
\hspace{8ex} \textit{Your Code Here} ;   // Count a "tail".
                 
\hspace{4ex} \}

\}
          
System.out.println("There were " + \textit{Your Code Here} + " heads.");
          
System.out.println("There were " + \textit{Your Code Here} + " tails.");
          
 \section*{Question 5}
 For this following question, we are going to instantiate a new monkey. And your task is to see what is the out put. Assume the Monkey class is already implemented. It takes a size and how many bananas it eats.\\
 
  Here it is: \\
  
 public static void main(String[] args) \{ 
 
 \hspace{4ex} Monkey champ = new Monkey(10, 200);
          
 \hspace{4ex} Monkey biggerchamp = champ;
 
 \hspace{4ex} biggerchamp.size = 40; 
 
 \hspace{4ex} System.out.println(champ.size); 
 
 \} \\ \newline What is the output? Does this change the size of the champ monkey?\\
          


\end{document}  